%% abtex2-modelo-slides.tex, v-1.0 gfabinhomat
%% Copyright 2012-2016 by abnTeX2 group at http://www.abntex.net.br/ 
%%
%% This work may be distributed and/or modified under the
%% conditions of the LaTeX Project Public License, either version 1.3
%% of this license or (at your option) any later version.
%% The latest version of this license is in
%%   http://www.latex-project.org/lppl.txt
%% and version 1.3 or later is part of all distributions of LaTeX
%% version 2005/12/01 or later.
%%
%% This work has the LPPL maintenance status `maintained'.
%% 
%% The Current Maintainer of this work is Fábio Rodrigues Silva, 
%% member of abnTeX2 team, led by Lauro César Araujo. 
%% Further information are available on 
%% http://www.abntex.net.br/
%%
%% This work consists of the files abntex2-modelo-slides.tex, 
%% abntex2-modelo-references.bib and abntex2-modelo-marca.pdf
%%
%% Modelo desenvolvido por Fábio Rodrigues Silva (gfabinhomat@gmail.com)
%% Mais informações podem ser obtidas no guia do usuário Beamer 
%% (http://linorg.usp.br/CTAN/macros/latex/contrib/beamer/doc/beameruserguide.pdf)
%% Informações rápidas podem ser acessadas em http://en.wikibooks.org/wiki/LaTeX/Presentations


% Apresentações em widescreen. Outros valores possíveis: 1610, 149, 54, 43 e 32.
% Por padrão, as apresentações são no formato 4:3 (sem o aspectratio).
%\documentclass[aspectratio=169, xcolor=table]{beamer}
\documentclass[aspectratio=169]{beamer}
%[aspectratio=169]
\usetheme{Pittsburgh}
\usecolortheme{default}
\usefonttheme[onlymath]{serif}			% para fontes matemáticas
% Enconte mais temas e cores em http://www.hartwork.org/beamer-theme-matrix/ 
% Veja também http://deic.uab.es/~iblanes/beamer_gallery/index.html

% Customizações de Cores: fg significa cor do texto e bg é cor do fundo
\setbeamercolor{normal text}{fg=black}
\setbeamercolor{alerted text}{fg=red}
\setbeamercolor{author}{fg=blue}
\setbeamercolor{institute}{fg=blue}
\setbeamercolor{date}{fg=green}
\setbeamercolor{frametitle}{fg=red}
\setbeamercolor{framesubtitle}{fg=brown}
\setbeamercolor{block title}{bg=blue, fg=white}		%Cor do título
\setbeamercolor{block body}{bg=gray, fg=darkgray}	%Cor do texto (bg= fundo; fg=texto)

\setbeamertemplate{navigation symbols}{ %\insertslidenavigationsym\\
\insertframenavigationsymbol \insertsubsectionnavigationsymbol \insertsectionnavigationsymbol \insertdocnavigationsymbol \insertbackfindforwardnavigationsymbol \hspace{1em} 	\usebeamerfont{footline}
\insertframenumber/\inserttotalframenumber%
}

% ---
% PACOTES
% ---
\usepackage[alf]{abntex2cite}		% Citações padrão ABNT
\usepackage[english]{babel}		% Idioma do documento
\usepackage{color}			% Controle das cores
\usepackage[T1]{fontenc}		% Selecao de codigos de fonte.
\usepackage{graphicx}			% Inclusão de gráficos
\usepackage[utf8]{inputenc}		% Codificacao do documento (conversão automática dos acentos)
\usepackage{txfonts}			% Fontes virtuais
%\usepackage{listings,bera}
\usepackage[portugues,ruled,lined]{algorithm2e}
\usepackage{algorithmic}
\usepackage{mathtools} % loads amsmath
\usepackage[absolute,overlay]{textpos}
\usepackage{tikz}
\usetikzlibrary{shadows}
%\usepackage[noend]{algpseudocode}
\usepackage{algorithmic}
\setbeamertemplate{caption}[numbered]

\usepackage{subcaption}
\usepackage{listings}
\usepackage{adjustbox}

% ---

% --- Informações do documento ---
\title{Radio Communication to Control and Run an Autonomous Mission for UAVs via a Mobile Application}
\author{Thiago Rodrigo Félix Cavalcante\\Erickson Higor da Silva Alves\\Celso Barbosa Carvalho}
%\institute{Programa de Pós-Graduação em Engenharia Elétrica
%	    \par
%	    Faculdade de Tecnologia}
\date{Manaus, \today}
% ---

%**
% \PutAt<overlay spec>[<box width>]{(<x>, <y>)}{<content>}
%
% real absolute positioning of <content> on a slide, if content is a figure,
% minipage or whatever kind of LR-box, the <box width> argument may be omitted
%
%
% implementation notes: 
%   - based on   \usepackage[absolute,overlay]{textpos}
%   - NOT combinable with any beamer feature that is based on pgfpages
%     (such as dual-screen support, built-in 2up handouts, etc.), as textpos 
%     and pgfpates interfere at the shippout-level.
%

  \newcommand<>{\PutAt}[3][0pt]{%
    {\only#4{\begin{textblock*}{#1}#2%
      #3
    \end{textblock*}}}%
  }

%**
% \ShowPutAtGrid
%
% draws a helpful grid on the current slide to figure <x> and <y> parameters for \PutAt
% 
  \newcommand{\ShowPutAtGrid}{
    \begin{textblock*}{128mm}(0cm,0cm)
    \tikz[color=red!20!white]\draw[very thin, step=5mm] (0mm,0mm) grid (130mm,100mm);
    \end{textblock*}
    \begin{textblock*}{128mm}(0cm,0cm)
    \begin{tikzpicture}[color=red]
      \draw[step=1cm] (0,0mm) grid (130mm,100mm);   
      \foreach \n in {0,...,12}
        \draw[xshift=.5mm,yshift=-1.5mm, inner sep=0pt, anchor=west] (\n,10) node {\scriptsize{\textbf{\n}}};
      \foreach \n in {1,...,9}
        \draw[xshift=.5mm,yshift=-1.5mm, inner sep=0pt, anchor=west] (0,10-\n) node {\scriptsize{\textbf{\n}}};
    \end{tikzpicture}
    \end{textblock*}
  }


%**
% \NormalBox<overlay spec>[tikz picture/node options]{<content>}
%
% draws content boxed in a nice box
% 
\newcommand<>{\NormalBox}[2][]{%
  \only#3{\tikz[#1, every node/.style={shape=rectangle,draw,fill=white, drop shadow, #1}]\node []{#2};}
}
%**
% \OrangeBox<overlay spec>[tikz picture/node options]{<content>}
%
% draws content boxed in an orange call-out box
% 
\newcommand<>{\OrangeBox}[2][]{%
  \onslide#3{\NormalBox[fill=orange!30,draw=black!30,rounded corners=4pt,#1]{#2}}%
}


% ----------------- INÍCIO DO DOCUMENTO --------------------------------------
\begin{document}

% ----------------- NOVO SLIDE --------------------------------
\begin{frame}

\begin{minipage}{1\linewidth}
  \centering
  \begin{tabular}{ccc}
    \begin{tabular}{c}
      \includegraphics[scale=0.13]{ufam.eps}
    \end{tabular}
    &
    \begin{tabular}{c}
      \textbf{Universidade Federal do Amazonas} \\ \textbf{Faculdade de Tecnologia}
    \end{tabular}
    &
    \begin{tabular}{c}
      \includegraphics[scale=0.3]{banner-verde_erin2017.png}
    \end{tabular}
  \end{tabular}
\end{minipage}

\titlepage

\end{frame}

% ----------------- NOVO SLIDE --------------------------------
\begin{frame}{Contents}
\tableofcontents
\end{frame}


% ----------------- NOVO SLIDE --------------------------------
\section{Introduction}

\begin{frame}{Introduction}

Logistics has become a competitive and fundamental factor for organizations, involving the management, conservation, and supervision of freight transport. In addition, excellent logistics means client satisfaction; so speed is still an important factor in a successful logistics process~\cite{drone4logistic}.

\end{frame}
\begin{frame}{Introduction}

\begin{itemize}
\item we present a methodology to carry out a mission, \textit{i.e.}, a client order, in a production environment. Such mission is previously planned, so that the distance travelled by the UAV is optimized;\pause
\item these components are implemented in an Android application, which uses radio waves to communicate with the drone;\pause
\item the mobile app is useful for leaders and/or managers, since they are able to supervise production lines in real-time, checking statuses and acting on failures.
\end{itemize}

\end{frame}

\section{Methodology}
\begin{frame}{Methodology}
\frametitle{Methodology}
\framesubtitle{Functioning Diagram}\pause

 \begin{figure}[H]
  \centering
  \includegraphics[scale=0.4]{sysArchAndPort.eps}
  %\caption{Diagram}
\end{figure}

\end{frame}

\begin{frame}{Methodology}
\frametitle{Methodology}
\framesubtitle{Use Case Environment}\pause

 \begin{figure}[H]
  \centering
  \includegraphics[scale=0.55]{useCase.eps}
  %\caption{Diagram}
 \end{figure}

\end{frame}

\section{Results}

\begin{frame}{Results}
\frametitle{Results} 	
%\framesubtitle{Sketching}
\pause

\begin{figure}[]
\centering
\begin{minipage}{.4\textwidth}
  \centering
  \includegraphics[width=.7\linewidth]{appMain.png}
  \captionof{figure}{Main Screen.}
  \label{fig:appMain}
\end{minipage}%
\begin{minipage}{.4\textwidth}
  \centering
  \includegraphics[width=.7\linewidth]{appSec.png}
  \captionof{figure}{Secondary Screen.}
  \label{fig:appSec}
\end{minipage}
\end{figure}

\end{frame}

\begin{frame}{Results}
\frametitle{Results} 	
%\framesubtitle{Sketching}
\pause

\begin{table}[H]
\centering
\caption{Mission Flight Time Tested in Real and Virtual UAV.}
\begin{tabular}{c|c|c|}
\cline{2-3}
\textbf{}                                     & \textbf{Real UAV}                 & \textbf{Virtual UAV}              \\ \hline
\multicolumn{1}{|c|}{\textit{\textbf{Tests}}} & \textit{\textbf{Flight Time (s)}} & \textit{\textbf{Flight Time (s)}} \\ \hline
\multicolumn{1}{|c|}{1}                       & 441.720                           & 430.830                           \\ \hline
\multicolumn{1}{|c|}{2}                       & 440.180                           & 436.885                           \\ \hline
\multicolumn{1}{|c|}{3}                       & 447.510                           & 441.681                           \\ \hline
\multicolumn{1}{|c|}{4}                       & 438.190                           & 441.227                           \\ \hline
\multicolumn{1}{|c|}{5}                       & 445.850                           & 451.865                           \\ \hline
\end{tabular}
\label{table:tests}
\end{table}

\end{frame}

\section{Conclusions}
\begin{frame}{Conclusions}
\frametitle{Conclusions}\pause

\begin{itemize}
\item we have developed an Android application to supervise production status and allow real-time monitoring for line managers/leaders;\pause

\item we have developed a framework for mission planning, command and control for intralogistics mission using a commercial UAV;\pause

\item we used an UAV to solve intralogistics problems using the DroneKit API to control and command adopting a high-level programming language.
\end{itemize}

\end{frame}

\section{Critique}
\begin{frame}{Critique}
\frametitle{Critique}\pause

\begin{itemize}
\item the use of computational vision for the recognition of inputs, and improvements of the UAV system used;\pause
\item evaluate the communication between the drone and the application using different frameworks/protocols, in order to improve the supervision step.
\end{itemize}

\end{frame}

% ----------------- NOVO SLIDE --------------------------------
\section{References}

% --- O comando \allowframebreaks ---
% Se o conteúdo não se encaixa em um quadro, a opção allowframebreaks instrui 
% beamer para quebrá-lo automaticamente entre dois ou mais quadros,
% mantendo o frametitle do primeiro quadro (dado como argumento) e acrescentando 
% um número romano ou algo parecido na continuação.

\begin{frame}[allowframebreaks]{References}
\bibliography{ref-slides}
\end{frame}

% ----------------- FIM DO DOCUMENTO -----------------------------------------
\end{document}
