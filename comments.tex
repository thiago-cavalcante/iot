\documentclass[12pt]{article}

\usepackage[normalem]{ulem}
\usepackage{xcolor}
\usepackage[english, portuguese]{babel}
\usepackage[T1]{fontenc}
\usepackage[utf8]{inputenc}

\begin{document}

\setcounter{secnumdepth}{0}

\section{Reviews}

\subsection{Reviewer \#1}

\subsubsection{Comments}

\sout{This paper shows the development of an Android application for controlling an unmanned aerial vehicle (UAV). The authors experiment providing a mission planner for carrying out  the manufacturing of products. The proposal is interesting and may be useful. However, it is not clear to me how the Android application was useful to assist in the supervising process for line managers/leaders. The real contribution of this paper was also not stated clearly. I think the paper formatting is wrong in the sense of lack of numbering in sections and subsections. Figures 6 and 7 are not referenced.} \textcolor{red}{Resolvido em outros comentários}

\begin{itemize}
\item \sout{Figures 6 and 7 are not referenced.}
\end{itemize}

\subsection{Reviewer \#2}

\subsubsection{Comments}

O artigo apresenta um aplicativo Android para controlar um VANT na tarefa de supervisão de uma linha de produção.

Pontos fortes:

\begin{itemize}
\item O artigo está bem estruturado, descreve claramente o problema motivador, bem como a solução proposta e os experimentos realizados.
\end{itemize}

Pontos fracos:

\begin{itemize}

\item \sout{O artigo não deixa bem claro qual seu objetivo. Pelo título e pela leitura do artigo como um todo, percebe-se que o conteúdo do trabalho enfoca igualmente a arquitetura do sistema, bem como o aplicativo Android, um dos componentes. Contudo, o Resumo e a Introdução deixam a entender que o foco do artigo é apenas o aplicativo, sendo que a descrição deste é relagado a pouco mais de um página e duas capturas de tela. Ao meu ver, o objetivo do artigo é mais amplo, como transparece pela leitura integral.}

\item A descrição da solução mistura-se com o estudo de caso em que foi testada. Por exemplo, creio que os inputs podem ser mais do que três (A, B e C) e os produtos podem ser mais do que dois (X e Y). Entendo que essa configuração atende bem o experimento realizado, mas ela não pode limitar a definição dos componentes da arquitetura.

\item \sout{A revisão bibliográfica é superficial, o que compromete ter uma referência para entender se os resultados obtidos no experimento e exibidos na Tabela 2 são razoáveis ou não.} \textcolor{red}{Resolvido em outros comentários}

\item \sout{O artigo deve se adequar ao modelo SBC, que numera as seções e subseções.} \textcolor{red}{Resolvido em outros comentários}

\end{itemize}

Sugestões:

\begin{itemize}

\item \sout{Na referências, detalhar o trabalho de Meier et al. (2015a). Trata-se de um artigo, URL, relatório técnico, outro?}

\item \sout{As legendas da tabelas devem ser posicionadas acima, e não abaixo, como no caso das Figuras.}

\item \sout{Converter as Tabelas 1 e 2 em texto, como a Tabela 3, pois elas ficaram desproporcionais em relação ao restante do documento.}

\item \sout{Como o artigo foi escrito em inglês, deve-se usar ponto (.) na Tabela 2 como separador de casas decimais.}

\item \sout{Revisar o endereço dos autores. Ter grau de mestrado não é um endereço.}

\item \sout{Pequenos aprimoramentos no texto em inglês: "Both academy and industry have done researches" -> have been researching; "the UAV executes to produce the clients order" -> accomplish the clients order; "we can see..." -> we show, we present.}

\end{itemize}

\subsection{Reviewer \#3}

\subsubsection{Comments}

\begin{itemize}
\item O trabalho está muito bem escrito e organizado. A aplicação desenvolvida apresenta um bom domínio técnico, apenas senti falta de alguma avaliação experimental mais aprofundada.

\end{itemize}

\end{document}